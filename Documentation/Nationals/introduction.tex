\chapter*{Introduction}
	\label{chap:introduction}

    %\addcontentsline{toc}{section}{Documentation Goals}
    %\section*{Documentation Goals}
    %This content was moved to Abstract

    \addcontentsline{toc}{section}{Accompanying Resources}
    \section*{Accompanying Resources}
    	\label{sec:resources}
        This documentation can stand by itself, however your understanding will be enhanced by using the extra content as outlined in the following subsections.

        \addcontentsline{toc}{subsection}{Wiki}
        \subsection*{Wiki}
        	\label{subsec:Wiki}
            \index{wiki}
            I've very recently started maintaining a wiki, which I hope to populate with all the knowledge that I've amassed through trial, failure, and success - not only from this year, but also from previous years of competitions.\\
            
            In early 2014 I received an email from someone who I had met at RoboCup Nationals, 2012 and given my email to. I couldn't remember the particular person, but they sent me an email asking about how to build custom built rescue robots. What ensued was a some 9620 word correspondence over 5 months, in which I outlined all the lessons that I had learned in regards to motors, locomotion, sensors, electronics, manufacturing, debugging, etc.\\
            
            \index{rescue}
            I don't claim to have a great deal of knowledge because I've succeeded a lot - I claim to have a small fortune of knowledge because I've failed so much. I know what doesn't work - because I've often experienced failure. There are, however, a few small successes - which over the course of 3 years shaped how I created my robots. The result was a robot that was almost "perfect" in 2013 - if I had continued Rescue for another year, I believe that I would've produced an almost perfect rescue robot.\\
            
            Because of the knowledge of all these failures, and how much time and money this knowledge could save if it was available to the general community of RoboCup Rescue, I thought of releasing this information on a Wiki. At this point it's in it's very early days - but I hope to continue development of it.\\
            
            \index{sensors}            
            I genuinely believe that a combination of my experiences, with the experiences of others culminated onto a single Wiki could greatly thrust the level of Rescue light years ahead. Imagine if a school could read a wiki which told them what sensors to buy, and what sensors not to buy - and listed all the reasons why.\\
            
            But not only would it lift the level of competition within RoboCup, it would in the end produce better engineers entering university, as they've been able to cover much more ground, and experience many more technologies than previously available.\\
            
            I've been unable to put a whole lot of content in the wiki before Nationals, 2015 - however I intend to put all my acquired knowledge in it over time. You can see the latest changes on the wiki on this page:\\
            
			\url{http://linnode.johnrobboard.com/wiki/index.php/Special:RecentChanges}\\
		
		\addcontentsline{toc}{subsection}{GitHub}	
		\subsection*{GitHub}
			\index{GitHub}
			\index{repository}
			Throughout this document I make mention to my code. I frequently mention that it's located on BitBucket in a private repository due to the copyright nature of some of the content I'm using. I've made a copy of this repository, removed any sensitive information, and uploaded it to GitHub. You can find this repository here:\\
			
			\url{https://github.com/boar401s2/RCJA2015}

        \addcontentsline{toc}{subsection}{YouTube}
        \subsection*{YouTube}
            Although I dearly love the communication medium of text, I have been learning to also love video. As such, I've been creating a few videos that help illustrate what I do in robotics. Although I only have a few robotics videos on there, and at fairly low quality - I intend to increase that all the time. You can check out my YouTube channel at:\\
            
            \url{https://www.youtube.com/channel/UC5_HY8gw3HC2Rq7axfU144g}\\            
        
        \addcontentsline{toc}{subsection}{Facebook}
        \subsection*{Facebook}
        \label{subsec:Facebook}
            I frequently share the status of what I do, with pictures and documentation on my Facebook account. Specifically written to the reviews of this logbook, should you wish to see further information, feel free to add me:\\
            
            \url{https://www.facebook.com/profile.php?id=100005747055401}\\
            
            \textbf{NOTE:} I am in the process of creating a public Facebook page where I will share these posts, that will accompany my website. It will be called WhiteBoard (as my website is also called WhiteBoard).\\
            
        \addcontentsline{toc}{subsection}{Website}
        \subsection*{Website}
            Over the last few weeks (and as documented in the Log chapter), I have been working on a personal website/portfolio. It is my intention in time that I share what I would normally share on Facebook on a blog here. I would then link to it using the Facebook page.\\

    \addcontentsline{toc}{section}{How to Read}
    \section*{How to Read}
        I recognize that this document can be quite lengthy for the average reader, and as such I've added some features that will hopefully help you find what you are looking for quickly.\\
        
        \index{log}\index{index}
        You could either read this document as a book from start to end - in which case you would get the entire view of this project. You could alternately read just the Log chapter, which would give you an "as it happens" approach to what I do on a day to day level. Finally you could treat this book as a reference book, and search for particular problems (and/or solutions) using the \hyperref[index]{index}.\\        
        
        Although I've tried to make the \hyperref[index]{index} comprehensive, it won't be. Try looking for what you want using the table of contents first, 
        
        People have questioned why I included an \hyperref[index]{index}, and the reason is because I have personally found that having an \hyperref[index]{index} is very beneficial - especially in large documents. An index can cover specific information, whereas the Table of Contents only gives a broad gist.\\
        
        The index should be especially useful in the log section, where I list specific problems that aren't mentioned in the table of contents.\\

    \addcontentsline{toc}{section}{Revisions}
    \section*{Revisions}
        This document is the third revision in the series.\\
        
        \index{regionals}        
        The first revision was for RoboCup \index{regionals} Regionals. It was compiled on the July 23, 2015. The first revision was 29 pages long.\\
        
        The first revision was for RoboCup \index{states}States. It was compiled on the August 13, 2015. The first revision was 43 pages long.\\
        
        The third revision is this, designed for RoboCup \index{Nationals}Nationals. This document was last compiled on the \today. It's length is currently undetermined - check the end of the book for more information!\\

    \addcontentsline{toc}{section}{Contact}
    \section*{Contact}
    \label{sec:contact}
    \label{contact}
    \index{contact me}
        A large portion of this documentation was created on a hurried schedule, and as such I expect there to be at least a few mistakes. If you feel so obliged to report these problems, or if you have any questions in general, please feel free to contact me using the details below.\\
        
        Alternately, if you want a digital copy of this document, feel free to send me an email at:\\
		\index{email}	
		\index{contact}
		\textbf{Email: }\href{mailto:documentation@johnrobboard.com}{documentation@johnrobboard.com}