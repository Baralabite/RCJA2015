	\chapter{Mechanical}
		This chapter outlines mechanical information about the Lynxmotion A-Pod. This chapter is most likely the least filled in. It needs a lot of work.

		\section{Dimensions}
        	\index{dimensions}
            \index{sizes}
			When standing, the top of the top body plate is 185mm from the ground on which it stands.\\
			
			Each femur joint should be 120mm from the ground.\\
			
			The femur joint is 50mm away from the coxa on the local X axis.\\
            
            All this joint information was used to generate the 3D mathematical model used in the inverse kinematics.\\
                        
		\section{Kit Information}
        	\index{kit}
			We ordered the Lynxmotion A-Pod as a kit with the mechanical components and servos.\\
			
			\url{http://www.lynxmotion.com/c-154-a-pod.aspx}\\
			
			In the same order we also purchased the required batteries, PS2 controller, BotBoarduino, and SCC-32 Servo Controller.\\
			
			The servos used are HiTec HS-645.\\
			\pagebreak
            
        \section{Maintenance}
        	\index{maintenance}
        	Not a lot of maintenance has to be done, however the screws and nuts on the servo brackets have a tenancy to come undone. I have looked at a few solutions such as spring washers, loctite, etc - however for now I'm simply regularly making sure that the screws are tight.\\
            
        \section{Motor Burnout}
        	\index{motor burnout}
            \index{motor failure}
        	\label{motor_failure}
            \index{servo}
        	I have learned through bad experience that if the servo motor is told to move to a position that is physically obstructed, and left there for a while it will burn out.\\
            
            The telltale signs of burnout is typically that the motor is warm or hot, potentially a bad small, and the motor seems stiffer than usual when turned off, and the joint moved by hand.\\
            
            Prevention is better than cure - make sure that a servo is not strained for too long. If this does occur, the only fix is to replace the motor. The motor brand is:\\
            
            \textbf{Hitech MS-645MG}\\
            
        \section{Screw Sizes}
        	\index{sizes}
            \index{screws}
        	This section is incomplete. More information is required on the screw and nut sizes of the robot.\\
            
            