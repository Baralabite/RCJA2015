	\chapter{Story}
    	\label{Story}
		I'm a big believer in adding a story to most anything. I think stories engage the \index{audience}audience, and draw them in. It creates an emotional attachment to the performance, and to the actors. I like to employ stories and storytelling in things I teach, and performances I put on.\\
        
        The story of this performance is loosely modeled after the story of the Prodigal Son in Luke 15:11-12 (The Bible) - and typically the creation/salvation story outlined in the Bible.\\
        
        In the story, I'm the creator/father figure, and I create my son/child/robot. We are at first having fun, and enjoying each other's company - but as time goes by the robot gets annoyed, and pushes me away. The robot walks off, but through which time the father wishes that his son was with him all the while. The father doesn't care that his son did him wrong - his love transcends that sin.\\
        
        After a while, the robot realizes that he was better off, and happier with his creator. They both come back together again, and lived happily ever after!\\              