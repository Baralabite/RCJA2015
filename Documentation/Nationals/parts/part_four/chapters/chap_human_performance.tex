	\chapter{Human Performance}
    	From the outset I realized that a good performance will include good \index{human interaction}human interaction. As expressed in the Dance Performance score sheet, the interaction should not detract from the robot itself - but I believe a performance can be rounded off with a good human performance.\\
        
        I'm naturally not a person who defaults to dancing or acting on stage - however because of requirement to do so, I recognized that I needed to put aside my illogical fears, and perform. The result of this is that I made an agreement with myself that I would put aside any fears or doubts about performing on stage, and that I was just do what I had to do, and put my all into it.\\
        
        I believe that because I made this agreement with myself, I was able to give a half decent performance, which compliments the robot. As I've mentioned previously, I don't want to detract from robot's performance, but at the same time I try to create a human-robot interaction throughout the performance in the setting of a story to increase the entertainment factor.\\
        
        Often my performance comes from what I want the robot to do. Often the comical factor of the robot is it showing attitude to me. It's quite simple to express emotion in reaction to the robot. For example, I can look surprised when the robot roars at me. Just having the robot perform is only one side of a conversation - having a robot respond to the robot adds the second side of the conversation.\\
        
        \section{Human Interaction Incidence}
        	\index{human_interaction}
        	There was one incident at State titles which caused a problem I hadn't anticipated. The issue was that how my routine was developed, I would stand to the left of the robot for the duration of the routine - as the robot would face me at times throughout the routine (ex. roaring at me).\\
            
        	On the day, I didn't have a good way of solving this problem (except for standing on the other side of the performance, and hope for the best), which is what I did.\\
            \index{judges}
            For \index{nationals}Nationals I'm conscious of this problem, but as of yet haven't found a suitable solution. The best solution is to create two instances of the performance - one for right side, and one for left. Unfortunately this effectively doubles my work, which I may not have time to do. I have a theory that I \textit{may} be able to "mirror" the actions in blender, and export the mirror, but I haven't been able to test this yet.\\
            
            For Nationals I will advise the judges previously to the nature of the positioning of the \index{routine}routine, and also I'll try to keep out of their view as much as possible.\\
            
			\section{Stage Presence/Audience Interactivity}
    	Having good stage presence can also really round out a routine. Having stage presence means that you are able to command and hold the attention of the audience, simply by your demeanor and confidence. This can be achieved by confidently introducing your routine, and talking directly to the audience.\\
        
        Audience Interactivity can include asking the audience to clap along, sing, or maybe even dance! This helps draw the \index{audience}audience in, and creates a fun and vibrant atmosphere. I employed asking the audience to clap along for \index{regionals}Regionals and States. Because I'm doing a different kind of routine for Nationals, I'm not sure if I am able to include this kind of interactivity.\\
        
        
            
            
        
        